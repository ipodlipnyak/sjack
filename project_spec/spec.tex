\documentclass{scrreprt}
%\documentclass{report}
\usepackage[utf8]{inputenc}
\usepackage[russian]{babel}
\usepackage{listings}
\usepackage{underscore}
%\usepackage[bookmarks=true]{hyperref}
\usepackage[colorlinks=true,linkcolor=blue,urlcolor=black,bookmarksopen=true]{hyperref}
\usepackage{enumerate}
\usepackage{xcolor}

\usepackage{tikz}
\usetikzlibrary{mindmap,trees}

\newcounter{magicrownumbers}
\newcommand\rownumber{\stepcounter{magicrownumbers}\arabic{magicrownumbers}}

\hypersetup{
    bookmarks=false,    % show bookmarks bar?
    pdftitle={Software Requirement Specification},    % title
    pdfauthor={Yiannis Lazarides},                     % author
    pdfsubject={TeX and LaTeX},                        % subject of the document
    pdfkeywords={TeX, LaTeX, graphics, images}, % list of keywords
    colorlinks=true,       % false: boxed links; true: colored links
    linkcolor=blue,       % color of internal links
    citecolor=black,       % color of links to bibliography
    filecolor=black,        % color of file links
    urlcolor=purple,        % color of external links
    linktoc=page            % only page is linked
}%
\def\myversion{1.0 }
\title{%
\flushright
\rule{8cm}{1.5pt}\vskip1cm
\Huge{SOFTWARE REQUIREMENTS\\ SPECIFICATION}\\
%\vspace{1.5cm}
%for\\
%\vspace{1.5cm}
%Materials Ordering System\\
\vspace{1.5cm}
\LARGE{Release 1.0\\}
%\vspace{1.5cm}
%\LARGE{Version \myversion approved\\}
%\vspace{1.5cm}
%Prepared by Yiannis Lazarides\\
%\vfill
%\rule{8cm}{1.5pt}
}
\date{}
%\usepackage{hyperref}
\begin{document}
\maketitle
\tableofcontents
\chapter{Структура проекта}
%\begin{quote}
Библиотеки, фреймворки и пр. используемое в проекте
%\end{quote}
%\section{Section two}

\begin{enumerate}[]
	\item Routing, Api:
	\begin{enumerate}[-]
	\item Slim micro-framework \\
		\url{https://www.slimframework.com/}
	\end{enumerate}

	\item Basic access authentication:
	\begin{enumerate}[-]
		\item Slim-basic-auth middleware \\
		\url{https://github.com/tuupola/slim-basic-auth}
	\end{enumerate}

	\item ORM, Event dispatcher:
	\begin{enumerate}[-]
		\item Illuminate \\
		\url{https://github.com/illuminate/database} \\
		\url{https://laravel.com/docs/5.7/eloquent}
	\end{enumerate}

	\item Mailer:
	\begin{enumerate}[-]
		\item PHPMailer \\
		\url{https://github.com/PHPMailer/PHPMailer}
	\end{enumerate}

	\item Templating engine:
	\begin{enumerate}[-]
		\item Twig \\
		\url{https://twig.symfony.com/}
	\end{enumerate}

	\item Frontend:
	\begin{enumerate}[-]
		\item AdminLTE \\
		\url{https://github.com/almasaeed2010/AdminLTE}
		\item Webpack bundler \\
		\url{https://webpack.js.org/}
		\item Vue.js framework \\
		\url{https://vuejs.org/}
	\end{enumerate}

	\item Datamining, Web-Sites Monitoring:
	\begin{enumerate}[-]
		\item Requests \\
		\url{http://docs.python-requests.org}
%		\item SQLAlchemy \\
%		\url{https://www.sqlalchemy.org/}
	\end{enumerate}
\end{enumerate}
\chapter{Общее описание проекта}
Скрипт на питоне будет проверять состояние сайтов из списка урлов в базе, отправляя им http-реквесты. 
Получение списка страниц для проверки и сохранение ответот производится соответствующими реквестами к RESTful api административной панели.
Скрипт стартует крон-задачей. За рассылку http-реквестов отвечает библиотека Requests. 
%Работу с базой данных обеспечивает ORM SQLAlchemy.
Данные проверяемых страниц хранятся в таблице SMPages.

Для административной панели используется микрофреймворк Slim. Он особенно удобен при создании RESTful api и в будущем его возможности можно будет использовать для координации микросервисов. 
Особенно - если вдруг потребуется создание демонов на внешних серверах (их нельзя разворачивать на бегете).

Для авторизации запросов используется HTTP Basic authentication. Все запросы к api также проходят эту аутентификацию. 
Из-за того что передача авторизационных данных идёт открытым текстом необходимо включить https с принудительной переадресацией.

Административная панель содержит следующие формы:
\begin{enumerate}
	\item Общие настройки - почтовый адрес для уведомлений, доменное имя сайта, чекбокс для включения https при авторизации
	\item Список сайтов - просмотр статусов сайтов. 
		Если та или иная страница сайта имеет статус отличный от 200, 
		статус сайта будет подсвечен соответсвующим цветом. 
		Есть возможность удалить все страницы сайта и отредактировать имя домена
	\item Список страниц - просмотр статусов страниц с сортировкой по статусу и имени, 
		редактирование и удаление страниц из списка, 
		фильтр для поиска страниц по имени или урлу.
		Есть возможность отметить и удалить группу страниц.
	\item Добавление новой страницы - url, домен сайта, имя/title страницы, использование https. 
	%	Опционально - тип авторизации(для начала ограничимся HTTP Basic auth), логин и пароль
	\item Массовый импорт - текстовое поле для построчного перечисления страниц для импорта. 
		В конце импорта обнаруженные ошибки в импортируемых данных будут отображены в этом же текстовом поле
\end{enumerate}

В списке страниц статусы будут подсвечиваться красным (404), оранжевым(301,302) и зеленым(200). 
Если у сайта есть страницы с красным или оранжевым статусом он будет подсвечен соответствующим цветом. Красный в приоритете над оранжевым.
В случае переадресации можно будет просмотреть конечный адрес.

RESTful api позволяет просматривать (в ответ прийдет application/json) и изменять состояние сайтов из списка в базе.
В момент обновления состояния той или иной страницы, выстреливает событие. Если новый статус ответа 404 - будет отправлено email уведомление на почтовый ящик администратора.

%Для модуля авторизации в структуре таблиц базы данных заложена возможность реализации role-based access control. 
%Для модуля авторизации заложена возможность реализации role-based access control. 
%Каждая группа пользователей имеет ограничение по списку доступных к посещению url-адресов. Соответствие "url:группа" хранится в таблице Permissions. Если соответствие не задано, адрес доступен всем.
Административная панель разбита на зоны доступа (\href{https://tools.ietf.org/html/rfc7235#section-2.2}{ Realm в терминологии HTTP Basic auth}). При переходе между ними потребуется подтведить авторизацию. 
Соответствие ``зона доступа : пользователь'' хранится в таблице Permissions.
В таблице Realms хранится имя зоны доступа и входящие в неё url-адреса.

\chapter{Этапы разработки}

\begin{tabular}{ccl}
\hline
№ & Время(ч) & Описание этапа \\
\hline
	\rownumber & 2 & Разработка структуры таблиц БД\\
	\rownumber & 6 & Создание структуры проекта и настройка основных модулей\\
	\rownumber & 2 & Создание шаблона главной темы административной панели\\
	\rownumber & 2 & Форма общих настроек\\
	\rownumber & 3 & Форма просмотра списка сайтов\\
	\rownumber & 5 & Форма просмотра списка страниц\\
	\rownumber & 2 & Форма добавления страницы\\
	\rownumber & 5 & Форма массового импорта\\
	\rownumber & 4 & Скрипт для проверки состояния страниц\\
	\rownumber & 6 & Деплой и тестирование на боевом сервере\\
	\hline \hline
	$\Sigma$ & 37 & \\
\hline
\end{tabular}

\chapter{Mindmap}
\definecolor{orange}{HTML}{f5f5cc}
\definecolor{green}{HTML}{199600}
\pagecolor{orange}
\begin{tikzpicture}
  \path[mindmap,
	concept color=red!40!orange,text=orange,
	,
	%concept/.append style={concept color=blue!80,minimum size=3.5cm},
	grow cyclic,
	level 1/.append style={level distance=4.8cm,sibling angle=120,minimum size = 2.5cm},
	level 2/.append style={level distance=3.3cm,sibling angle=45,minimum size = 2cm},
	level 3/.append style={level distance=2.5cm,sibling angle=45,minimum size = 1.5cm},
	level 4/.append style={level distance=2cm,sibling angle=45,minimum size = 1cm},
	every annotation/.style={fill=black!20,scale=1,text width=3cm,
	text=black,
	%inner sep = 4mm,
	outer sep=15mm},
	]
	node[concept] (root) {Web-app}
    [clockwise from=0]
    child[concept color=blue!40!orange] {
      node[concept] {Dashboard}
      [clockwise from=90]
      child { node[concept] {Настройки} }
	child{
		node[concept] 
		{Проверка сайтов}
		child{node[concept]{Список сайтов}}
		child{node[concept]{Список страниц}}
		child{node[concept]{Новая страница}}
		child{node[concept]{Массовый импорт}}
		}
%      child { 
%	node[concept] 
%	{Инструменты}
%	child{
%		node[concept] 
%		{Мониторинг сайтов}
%		child{node[concept]{Список страниц}}
%		child{node[concept]{Добавление новой страницы}}
%		}
%	}
    } 
    child[concept color=green!40!orange] {
      node[concept] {RESTful Api}
      [clockwise from=-90]
      child { node[concept] {Update page} }
      child { node[concept] {Get all pages} }
    } 
    child[concept color=black!40!orange] {
	    node[concept] (sql) {MySQL}
	    [clockwise from=225]
	    child { node[concept] {Config} }
      child { node[concept] {Users} }
      child { node[concept] {Realms} }
      child { node[concept] {Permissions} }
      child { 
	node[concept] (pages) {SMPages} 
	node[annotation,right] at (pages) {
		SMPages table:
		\begin{enumerate}[-]
			\setlength\itemsep{-0.5em}
			\item	id,
			\item	page name,
			\item	site domain,
			\item	page url,
			\item	http/https
%			\item	auth login, 
%			\item	auth pass 
		\end{enumerate}
		}
	}
    }; 
\end{tikzpicture}

% add other chapters and sections to suit
\end{document}
